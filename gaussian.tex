\section{The Fourier Transform of a Gaussian}

Consider the signal

\begin{equation}
  \label{eq:gauss}
  x(t) = \frac{1}{\sqrt{\pi \sigma^2}} e^{-t^2/\sigma^2}, \quad t \in [-T/2,T/2)
\end{equation}

where $\sigma = 16 \Delta t$, step length $\Delta t = 1/N$ and $N = 1024$ samples. The signal is shown in Fig. \ref{fig:gaussSignal}. Doing a FFT on this signal gives a restriction on the maximum frequency one can obtain. This is given by the Nyqvist frequency

\begin{equation*}
  \nu_c = \frac{1}{2\Delta t},
\end{equation*}

which in our case is $\nu_c = 512$Hz. The frequency resolution is

\begin{equation*}
  f_{res} = \frac{f_{samp}}{N}
\end{equation*}

which in our case is equal to $1\,$Hz since the sampling rate is $1024\,$Hz.
 The FFT is plotted in Fig. \ref{fig:FFTgauss} and normalized together with the analytical fourier transform of the signal. The numerical solution correspond well to the analytical result with a maximum error of $4.86 \cdot 10^{-10}$ in the absolute norm.

\begin{figure}[H]
  \centering
  \includegraphics[width=0.8\textwidth]{figs/gaussianSignal.pdf}
  \caption{Gaussian signal of 1024 samples.}
  \label{fig:gaussSignal}
\end{figure}


\begin{figure}[H]
  \centering
  \includegraphics[width=0.8\textwidth]{figs/FFTgauss.pdf}
  \caption{FFT of the signal in Fig. \ref{fig:gaussSignal}.}
  \label{fig:FFTgauss}
\end{figure}
