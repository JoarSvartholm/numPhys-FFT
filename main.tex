\documentclass[a4paper,11pt]{article}
\usepackage[utf8]{inputenc}            % Tekenkodning
\usepackage[T1]{fontenc}               % Fixa kopiering av texten
\usepackage[english]{babel}            % Språk (t.ex. Innehåll)
\usepackage{geometry}                  % Sidlayout m.m.
\usepackage{graphicx,epstopdf,float}   % Bilder
\usepackage{amsmath,amssymb,amsfonts}  % Matematik
\usepackage{enumerate}                 % Fler typer av listor
\usepackage{fancyhdr}                  % Sidhuvud/sidfot
\usepackage{hyperref}                  % Hyperlänkar
\usepackage{parskip}                   % noindent!!
\usepackage{float}                     % \begin{figure}[H] preciserar bildposition
\usepackage{subcaption}             % För att lägga figurer bredvidvarandra från: http://tex.stackexchange.com/questions/91224/placing-two-figures-side-by-side
\usepackage{textcomp}
\usepackage{gensymb}
\usepackage{wrapfig}
\usepackage[]{algorithm2e}
% packages for matlab codes
\usepackage{listings}
\usepackage{color}
\usepackage{pdflscape}
\usepackage{multicol}
\setlength{\columnsep}{0.6cm}
% Add new commands used-defined:
\newcommand{\m}[1]{\mathbf{#1}}
\newcommand{\tn}[1]{\textnormal{#1}}
\newcommand{\ve}[1]{\textnormal{vec}(#1)}


% instälningar för figurtexter
%\usepackage[margin=3ex,font=small,labelfont=bf,labelsep=endash]{caption}
\usepackage[font={small,it}]{caption}
\usepackage[labelfont={normal,bf}]{caption}
\usepackage[margin=3ex]{caption}

% mailadresser som hyperlänkar
\newcommand{\mail}[1]{\href{mailto:#1}{\nolinkurl{#1}}}
% Spara författare och titel
\let\oldAuthor\author
\renewcommand{\author}[1]{\newcommand{\myAuthor}{#1}\oldAuthor{#1}}
\let\oldTitle\title
\renewcommand{\title}[1]{\newcommand{\myTitle}{#1}\oldTitle{#1}}

% Hyperlänkar
\hypersetup{
  colorlinks   = true, %Colours links instead of ugly boxes
  urlcolor     = black, %Colour for external hyperlinks
  linkcolor    = black, %Colour of internal links
  citecolor   = black  %Colour of citations
}





\graphicspath{{./matlab/},{./images/},{./matlab/figs/}} % Söker också bilder i en undermapp figs.


%% DOCUMENT
%------------------------------------------------------------------%
\begin{document}
  \title{Fast Fourier Transforms}


  \author{
    Joar Svartholm - josv0150(\mail{josv0150@student.umu.se})\\
  }
  \date{\today}


\begin{titlepage}
  \maketitle
  \thispagestyle{fancy}
  \headheight 35pt
  \rhead{\small\today}
  \lhead{\small Department of Physics\\
    Umeå Universitet}



% State the aim of the experiment, what was measured, which techniques and methods were used, and the main result(s) and conclusion(s). Remember that the abstract should be understandable on its own, and you can thereby not refer to equations/figures/tables in the report. You should also not use references, since the information in the abstract should be available in the actual report.

  % Ändra till rätt namn m.m.
  \cfoot{Numerical Methods in Physics \\
  Supervisor: Claude Dion}

\end{titlepage}


\newpage
\pagestyle{fancy}
\headheight 30pt
\rhead{\small \myTitle\\\today}
\lhead{\small \myAuthor}
\cfoot{\thepage}

% Innehåll
\tableofcontents
\newpage


\section{The Fourier Transform of a Gaussian}

Consider the signal

\begin{equation}
  \label{eq:gauss}
  x(t) = \frac{1}{\sqrt{\pi \sigma^2}} e^{-t^2/\sigma^2}, \quad t \in [-T/2,T/2)
\end{equation}

where $\sigma = 16 \Delta t$, step length $\Delta t = 1/N$ and $N = 1024$ samples. The signal is shown in Fig. \ref{fig:gaussSignal}. Doing a FFT on this signal gives a restriction on the maximum frequency one can obtain. This is given by the Nyqvist frequency

\begin{equation*}
  \nu_c = \frac{1}{2\Delta t},
\end{equation*}

which in our case is $\nu_c = 512$Hz. The frequency resolution is

\begin{equation*}
  f_{res} = \frac{f_{samp}}{N}
\end{equation*}

which in our case is equal to $1\,$Hz since the sampling rate is $1024\,$Hz.
 The FFT is plotted in Fig. \ref{fig:FFTgauss} and normalized together with the analytical fourier transform of the signal. The numerical solution correspond well to the analytical result with a maximum error of $4.86 \cdot 10^{-10}$ in the absolute norm.

\begin{figure}[H]
  \centering
  \includegraphics[width=0.8\textwidth]{figs/gaussianSignal.pdf}
  \caption{Gaussian signal of 1024 samples.}
  \label{fig:gaussSignal}
\end{figure}


\begin{figure}[H]
  \centering
  \includegraphics[width=0.8\textwidth]{figs/FFTgauss.pdf}
  \caption{FFT of the signal in Fig. \ref{fig:gaussSignal}.}
  \label{fig:FFTgauss}
\end{figure}


\section{The spectrum of a simple AM wave}

Now if we instead look at a sinusoidal signal with a modulated amplitude, Fig. \ref{fig:amsignal}, where each period correspond to a bit, 0 or 1. This was done by sampling 1024 samples from

\begin{equation}
\label{eq:amwave}
  u(t) = \bar{u}\sin{(2\pi f_c t)}
\end{equation}

where $f_c = 1/(128\Delta t)$ is the center frequency of the signal and $\Delta t = 1/1024$. The modulation is done such that an amplitude $\bar{u} = 3$ correspond to 1 and $\bar{u} = 1$ correspond to 0.

\begin{figure}[h]
  \centering
  \includegraphics[width=0.8\textwidth]{figs/amsignal}
  \caption{Amplitude modulated signal.}
  \label{fig:amsignal}
\end{figure}

Computing the FFT of this signal gives Fig. \ref{fig:FFTam} where the absolute value $S(f) = |U(f)|^2$ is plotted for only the positive freguencies. Looking at the peaks in this figure one can find the center frequency as 8 Hz which is in accordance with Eq. \eqref{eq:amwave}. The band width of this signal is found by the side peaks at 4 and 12 Hz to be $\Delta f = 8 Hz$. This could have been deduced from the supercomposition of two waves with frequencies 4 and 12 Hz. The bandwidth is thus simply the beat frequency of this signal.

\begin{figure}[h]
  \centering
  \includegraphics[width=0.8\textwidth]{figs/FFTam}
  \caption{Fourier transform of the signal from Fig. \ref{fig:amsignal}. There are three peaks at 4, 8 and 12 Hz.}
  \label{fig:FFTam}
\end{figure}

Changing the amplitude from 3 to 10 in the modulation gives the frequency spectrum in Fig. \ref{fig:FFTam10}. As one can see the peaks are at the same place but with a larger amplitude. This means that a larger modulation makes it easier to find in the spectrum. This makes sense.

\begin{figure}[h]
  \centering
  \includegraphics[width=0.8\textwidth]{figs/FFTam10}
  \caption{Fourier transform with a larger amplitude of the modulation.}
  \label{fig:FFTam10}
\end{figure}

If the frequency is doubled but with the same bandwidth one can send twice as many bits. Such a signal is shown in Fig. \ref{fig:amsignal2}. This means that as long as the hardware can handle high frequencies it is always good to use higher frequencies when sending signals.

\begin{figure}[h]
  \centering
  \includegraphics[width=0.8\textwidth]{figs/amsignal2}
  \caption{Amplitude modulated signal with doubled frequency.}
  \label{fig:amsignal2}
\end{figure}
\clearpage


\section{Extracting information from a noisy signal}

A more real signal is shown in Fig. \ref{fig:noisySignal}. Because of all the noise, no bit structure is visual in the signal. The signal contains 8192 samples over 1 second but only a partition of it is plotted in the figure.

\begin{figure}[H]
  \centering
  \includegraphics[width=0.8\textwidth]{figs/noisySignal}
  \caption{Noisy unfiltered signal with no clear bit structure.}
  \label{fig:noisySignal}
\end{figure}

The frequency spectrum of the signal is shown in Fig. \ref{fig:FFTnoisy}. One can see three distinct peaks and two minor peaks on each side of these. From the three peaks the bandwidth was estimated to $\Delta f = 256$Hz and the center frequency to $f_c = 1024$Hz. To this spectrum a gaussian filter

\begin{equation}
  \label{eq:gaussfilt}
  H(f) = \text{exp}\left[-\frac{1}{2}\left(\frac{|f|-f}{\Delta f}\right)^2\right]
\end{equation}

was added in order to see the signal. Using the estimated parameters a nice and clear signal was obtained. This is shown in Fig. \ref{fig:filteredSignal}. From this each bit with a maximum amplitude larger than unity was read off as 1 and the other bits as 0. The hidden message in the signal was found to be \verb|FNM18 'Ef!Z4h#7_| using the \verb|am11.dat| file.

\begin{figure}[H]
  \centering
  \includegraphics[width=0.8\textwidth]{figs/FFTnoisy}
  \caption{Frequency spectrum of the signal in Fig. \ref{fig:noisySignal}.}
  \label{fig:FFTnoisy}
\end{figure}

\begin{figure}[H]
  \centering
  \includegraphics[width=0.8\textwidth]{figs/filteredSignal}
  \caption{Filtered signal. A clear bit structure is now visible.}
  \label{fig:filteredSignal}
\end{figure}


\section{End words}

All C code is found in the code folder;\newline \verb|/home/josv0150/Documents/numPhys/numPhys-FFT/code| including the code for plotting. If something needs to be recompiled simply type \verb| make gaussian| for the gaussian.c file etc. The plot functions are written in python3 and the line \verb|python3 plotDecoder.py| or equivalent should do the trick. I have not tested to run these codes on sesam since I uploaded the codes remotely and I have not tried it for other python versions either.



\end{document}
