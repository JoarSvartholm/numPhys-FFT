\section{The spectrum of a simple AM wave}

Now if we instead look at a sinusoidal signal with a modulated amplitude, Fig. \ref{fig:amsignal}, where each period correspond to a bit, 0 or 1. This was done by sampling 1024 samples from

\begin{equation}
\label{eq:amwave}
  u(t) = \bar{u}\sin{2\pi f_c t}
\end{equation}

where $f_c = 1/(128\Delta t)$ is the center frequency of the signal and $\Delta t = 1/1024$. The modulation is done such that an amplitude $\bar{u} = 3$ correspond to 1 and $\bar{u} = 1$ correspond to 0.

\begin{figure}[H]
  \centering
  \includegraphics[width=0.8\textwidth]{figs/amsignal}
  \caption{Amplitude modulated signal.}
  \label{fig:amsignal}
\end{figure}

Computing the FFT of this signal gives Fig. \ref{fig:FFTam} where the absolute value $S(f) = |U(f)|^2$ is plotted for only the positive freguencies. Looking at the peaks in this figure one can find the center frequency as 8 Hz which is in accordance with Eq. \eqref{eq:amwave}. The band width of this signal is found by the side peaks at 4 and 12 Hz to be $\Delta f = 8 Hz$. This could have been deduced from the supercomposition of two waves with frequencies 4 and 12 Hz. The bandwidth is thus simply the beat frequency of this signal.

\begin{figure}[H]
  \centering
  \includegraphics[width=0.8\textwidth]{figs/FFTam}
  \caption{Fourier transform of the signal from Fig. \ref{fig:amsignal}. There are three peaks at 4, 8 and 12 Hz.}
  \label{fig:FFTam}
\end{figure}

Changing the amplitude from 3 to 10 in the modulation gives the frequency spectrum in Fig. \ref{fig:FFTam10}. As one can see the peaks are at the same place but with a larger amplitude. This means that a larger modulation makes it easier to find in the spectrum. This makes sense.

\begin{figure}[H]
  \centering
  \includegraphics[width=0.8\textwidth]{figs/FFTam10}
  \caption{Fourier transform with a larger amplitude of the modulation.}
  \label{fig:FFTam10}
\end{figure}

If the frequency is doubled but with the same bandwidth one can send twice as many bits. Such a signal is shown in Fig. \ref{fig:amsignal2}. This means that as long as the hardware can handle high frequencies it is always good to use higher frequencies when sending signals.

\begin{figure}[H]
  \centering
  \includegraphics[width=0.8\textwidth]{figs/amsignal2}
  \caption{Amplitude modulated signal with doubled frequency.}
  \label{fig:amsignal2}
\end{figure}
