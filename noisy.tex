\section{Extracting information from a noisy signal}

A more real signal is shown in Fig. \ref{fig:noisySignal}. Because of all the noise, no bit structure is visual in the signal. The signal contains 8192 samples over 1 second but only a partition of it is plotted in the figure.

\begin{figure}[H]
  \centering
  \includegraphics[width=0.8\textwidth]{figs/noisySignal}
  \caption{Noisy unfiltered signal with no clear bit structure.}
  \label{fig:noisySignal}
\end{figure}

The frequency spectrum of the signal is shown in Fig. \ref{fig:FFTnoisy}. One can see three distinct peaks and two minor peaks on each side of these. From the three peaks the bandwidth was estimated to $\Delta f = 256$Hz and the center frequency to $f_c = 1024$Hz. To this spectrum a gaussian filter

\begin{equation}
  \label{eq:gaussfilt}
  H(f) = \text{exp}\left[-\frac{1}{2}\left(\frac{|f|-f}{\Delta f}\right)^2\right]
\end{equation}

was added in order to see the signal. Using the estimated parameters a nice and clear signal was obtained. This is shown in Fig. \ref{fig:filteredSignal}. From this each bit with a maximum amplitude larger than unity was read off as 1 and the other bits as 0. The hidden message in the signal was found to be \verb|FNM18 'Ef!Z4h#7_| using the \verb|am11.dat| file.

\begin{figure}[H]
  \centering
  \includegraphics[width=0.8\textwidth]{figs/FFTnoisy}
  \caption{Frequency spectrum of the signal in Fig. \ref{fig:noisySignal}.}
  \label{fig:FFTnoisy}
\end{figure}

\begin{figure}[H]
  \centering
  \includegraphics[width=0.8\textwidth]{figs/filteredSignal}
  \caption{Filtered signal. A clear bit structure is now visible.}
  \label{fig:filteredSignal}
\end{figure}
